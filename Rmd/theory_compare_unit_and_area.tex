\subsection{From Unit to Area Level
Models}\label{from-unit-to-area-level-models}

In later simulation studies we will consider data in which area level
statistics are computed from individual information. From a contextual
point of view, starting from individual information is advantageous in
the sense that outlying areas can be motivated more easily. Also the
question for a good estimator for the sampling variances can be
motivated when knowing the underlying individual model. Hence, I will
derive the Fay-Herriot model starting from unit-level. Consider the
following model: \[
y_{\indexDomain\indexUnit} = \xUnit^\top\beta + \randomEffectIndexed + \samplingErrorUnitIndexed \text{ ,}
\] where $y_{\indexDomain\indexUnit}$ is the response in domain
$\indexDomain$ of unit $\indexUnit$ with
$\indexUnit = 1, \dots, \nUnitIndexed$, where $\nUnitIndexed$ is the
number of units in domain $\indexDomain$. $\randomEffectIndexed$ is an
area specific random effect following (i.i.d.) a normal distribution
with zero mean and $\randomEffectVariance$ as variance parameter.
$\samplingErrorUnitIndexed$ is the remaining deviation from the model,
following (i.i.d.) a normal distribution with zero mean and
$\samplingVariance$ as variance parameter. This unit level model is
defined under strong assumptions, still, assumptions most practitioner
are willing to make which could simplify the identification of the
sampling variances under the area level model.

From this model consider the area statistics
$\directStatIndexed = \frac{1}{\nUnitIndexed} \sum_{j = 1}^{\nUnitIndexed}y_{\indexDomain\indexUnit}$,
for which an area level model can be derived as: \[
\directStatIndexed = \xArea^\top\beta + \randomEffectIndexed + \samplingErrorIndexed
\] Considering the mean in a linear model, it can be expressed as
$\bar{y} = \bar{x}\beta$; the random effect was defined for each area,
hence it remains unaltered for the area level model. The error term in
this model can be expressed as the sampling error and its standard
deviation as the (conditional) standard deviation of the aggregated area
statistic, which in this case is a mean. Hence,
$\samplingErrorIndexed \sim N(0, \samplingVarianceIndexed = \samplingVariance / \nUnitIndexed)$.
Under this unit level model a sufficient estimator for
$\samplingVarianceIndexed$ can be derived from estimating
$\samplingVariance$, which can be done robust and non-robust in many
ways.
