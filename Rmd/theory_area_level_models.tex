\section{Area Level Models}\label{area-level-models}

Area Level Models in Small Area Estimation play an important role in the
production of reliable domain estimates.

\begin{itemize}
\itemsep1pt\parskip0pt\parsep0pt
\item
  They can be used even if unit level observations are not accessible.
\item
  In a model based estimation it is largely unsolved to incorporate
  design weights. Area level models can be used to start from a direct
  design based estimator.
\item
  Unit level models often have problems with heterogeneity. An
  assumption, for example, for unit level data is that the error terms
  of a model are homescedastic given the random effects. This assumption
  is often not plausible and may call for more complex assumptions on
  the variance structure of the data. However such structures may or may
  not be known and cannot be modelled easily. This can also lead to
  computationally demanding procedures.
\end{itemize}

Given these considerations the most important factor to choose cadidate
models is the availability of data. Very often there is not much of a
choice but rather a decission given the available information. And given
the availability of unit level data, the obvious choice is to consider a
model which can use such information. Only if that fails for one of the
above reasons can an area level model be of interest.

\subsection{The Fay Herriot Model}\label{the-fay-herriot-model}

A frequently used model in Small Area Estimation is a model introduced
by \textcite{Fay79}. It starts on the area level and is used in small
area estimation for research on area-level. It is build on a sampling
model: \[
\si{\tilde{y}} = \si{\theta} + \si{e},
\] where $\si{\tilde{y}}$ is a direct estimator of a statistic of
interest $\si{\theta}$ for an area $i$ with $i = 1, \dots, D$ and $D$
being the number of areas. The sampling error $\si{e}$ is assumed to be
independent and normally distributed with known variances $\sige$, i.e.
$\si{e}|\si{\theta} \sim \Distr{N}(0, \sige)$. The model is modified
with the linking model by assuming a linear relationship between the
true area statistic $\si{\theta}$ and some diterministic auxiliary
variables $\si{x}$: \[
\si{\theta} = \si{x}^\top \beta + \si{\re}
\] Note that $\si{x}$ is a vector containing area-level (aggregated)
information for $P$ variables and $\beta$ is a vector ($1\times P$) of
regression coefficients describing the (linear) relationship. The model
errors $\re$ are assumed to be independent and normally distributed,
i.e. $\re_i \sim \Distr{N}(0, \sigre)$ furthermore $e_i$ and $\re_i$ are
assumed to be independent. Combining the sampling and linking model
leads to:

\begin{align}
\label{eq:FH}
\tilde{y}_i = x_i^\top \beta + \re_i + e_i.
\end{align}
