\subsection{Fixed Point Algorithms}\label{fixed-point-algorithms}

Inspired by: Chatrchi Golshid (2012): Robust Estimation of Variance
Components in Small Area Estimation, Master-Thesis, Ottawa, Ontario,
Canada: p.~16ff.:

\begin{quote}
The fixed-point iterative method relies on the fixed-point theorem: "If g(x) is
a continuous function for all $x \in [a; b]$, then $g$ has a fixed point in $[a;
b]$." This can be proven by assuming that $g(a)\geq a$ and $g(b)\leq b$. Since
$g$ is continuous the intermediate value theorem guarantees that there exists a
$c$ such that $g(c) = c$.
\end{quote}

Starting from equation \ref{eq:rml_theta} where
$\theta = (\sigma_1^2,  \sigma_2^2)$ and $(\rho_1, \rho_2)$ are assumed
to be known, we can rewrite the equation such that:

\begin{multline}
\label{eq:rml_theta_fp}
\Phi_l(\theta) = \psi(\mat{r})^\top\mat{U}^{\frac{1}{2}}\mat{V}^{-1}
\frac{\partial\mat{V}}{\partial\theta_l}
\mat{V}^{-1}\mat{U}^{\frac{1}{2}} \psi(\mat{r}) -\\
\Tr{K\mat{V}^{-1}\frac{\partial\mat{V}}{\partial\theta_l}
(\mat{Z}\mat{V}_\re\mat{Z}^\top)^{-1} (\mat{Z}\mat{V}_\re\mat{Z}^\top)} = 0
\end{multline}

Note that because the matrix $\mat{V}_e$ is assumed to be known for the
FH model, it can be omitted. Let $\mat{0}_{r\times c}$ define a matrix
filled with $0$ of dimension $(r \times c)$ then:

\begin{align*}
\mat{Z}\mat{V}_u\mat{Z}^\top =& \mat{Z}\left(\begin{matrix}
\sigma_1^2\Omega_1 & \mat{0}_{D\times DT}\\
\mat{0}_{DT\times D} &  \sigma_2^2\Omega_2
\end{matrix}\right)\mat{Z}^\top \\
=& \mat{Z}
\left[
\sigma_1^2\left(\begin{matrix}
\Omega_1 & \mat{0}_{D\times DT} \\
\mat{0}_{DT\times D} &  \mat{0}_{DT\times DT}
\end{matrix}\right) + 
\sigma_2^2\left(\begin{matrix}
\mat{0}_{D\times D} & \mat{0}_{D\times DT} \\
\mat{0}_{DT\times D} & \Omega_2
\end{matrix}\right)
\right]\mat{Z}^\top \\
=& \left(\begin{matrix}\mat{Z}\bar{\Omega}_1\mat{Z}^\top & \mat{Z}\bar{\Omega}_2\mat{Z}^\top\end{matrix}\right)
\left(\begin{matrix}
\sigma_1^2 \\
\sigma_2^2
\end{matrix}\right)
\end{align*}

Thus equation \ref{eq:rml_theta_fp} can be rewritten to:

\begin{multline*}
\psi(\mat{r})^\top\mat{U}^{\frac{1}{2}}\mat{V}^{-1}\frac{\partial\mat{V}}{\partial\theta_l}\mat{V}^{-1}\mat{U}^{\frac{1}{2}} \psi(\mat{r}) = \\ \tr\left(K\mat{V}^{-1}\frac{\partial\mat{V}}{\partial\theta_l} (\mat{Z}\mat{V}_u\mat{Z}^\top)^{-1} \left(\begin{matrix}\mat{Z}\bar{\Omega}_1\mat{Z}^\top & \mat{Z}\bar{\Omega}_2\mat{Z}^\top\end{matrix}\right)\left(\begin{matrix}
\sigma_1^2 \\
\sigma_2^2
\end{matrix}\right)\right)
\end{multline*}

Let \[
\left(\begin{matrix}
\psi(\mat{r})^\top\mat{U}^{\frac{1}{2}}\mat{V}^{-1}\frac{\partial\mat{V}}{\partial\sigma_1^2}\mat{V}^{-1}\mat{U}^{\frac{1}{2}} \psi(\mat{r}) \\
\psi(\mat{r})^\top\mat{U}^{\frac{1}{2}}\mat{V}^{-1}\frac{\partial\mat{V}}{\partial\sigma_2^2}\mat{V}^{-1}\mat{U}^{\frac{1}{2}} \psi(\mat{r})
\end{matrix}\right)
= a(\theta) \text{ ,}
\] then \[
\theta = \left(\begin{matrix}
\sigma_1^2 \\
\sigma_2^2
\end{matrix}\right) = A(\theta)^{-1} a(\theta) \text{ ,}
\] where

\begin{align*}
A(\theta) = \left(\begin{matrix}
\tr\left(K\mat{V}^{-1}\frac{\partial\mat{V}}{\partial\sigma_1^2} (\mat{Z}\mat{V}_u\mat{Z}^\top)^{-1} \mat{Z}\bar{\Omega}_1\mat{Z}^\top \right) &
\tr\left(K\mat{V}^{-1}\frac{\partial\mat{V}}{\partial\sigma_1^2} (\mat{Z}\mat{V}_u\mat{Z}^\top)^{-1} \mat{Z}\bar{\Omega}_2\mat{Z}^\top \right) \\
\tr\left(K\mat{V}^{-1}\frac{\partial\mat{V}}{\partial\sigma_2^2} (\mat{Z}\mat{V}_u\mat{Z}^\top)^{-1} \mat{Z}\bar{\Omega}_1\mat{Z}^\top \right) &
\tr\left(K\mat{V}^{-1}\frac{\partial\mat{V}}{\partial\sigma_2^2} (\mat{Z}\mat{V}_u\mat{Z}^\top)^{-1} \mat{Z}\bar{\Omega}_2\mat{Z}^\top \right)
\end{matrix}\right) \text{.}
\end{align*}

So, the fixed point algorithm can be presented as follows: \[
\theta^{m+1} = A(\theta^{(m)})^{-1} a(\theta^{(m)})
\]

At this time the fixed-point algorithm for
$\theta = (\sigma_1^2, \sigma_2^2)$ will replace the corresponding step
in Issue 1.

\subsubsection{N-S: Fixed-Point-Algorithm - Spatial
Correlation}\label{n-s-fixed-point-algorithm---spatial-correlation}

To extend the above algorithm to not only being used for the estimation
of $\theta = (\sigma_1^2, \sigma_2^2)$ but also for the spatial
correlation parameter $\rho_1$ reconsider:

\begin{eqnarray}
\label{eq:zVuZ_rho1_1}
\mat{Z}\mat{V}_u\mat{Z}^\top &=& \mat{Z}\left(\begin{matrix}
\sigma_1^2\Omega_1 & \mat{0}_{D\times DT}\\
\mat{0}_{DT\times D} &  \sigma_2^2\Omega_2
\end{matrix}\right)\mat{Z}^\top
\end{eqnarray}

and the specification of
$\Omega_1(\rho_1) = \left((I-\rho_1\mat{W})^\top(I-\rho_1\mat{W})\right)^{-1}$:

\begin{align}
\sigma_1^2\Omega_1(\rho_1) &= \sigma_1^2\Omega_1\Omega_1(I-\rho_1\mat{W})^\top(I-\rho_1\mat{W}) \notag\\
&= \sigma_1^2\left(\Omega_1\Omega_1 -\rho_1\Omega_1\Omega_1\mat{W}^\top -\rho_1\Omega_1\Omega_1\mat{W} + \rho_1^2\Omega_1\Omega_1\mat{W}^\top\mat{W}\right) \notag\\
&= \sigma_1^2\left(\Omega_1\Omega_1 -\rho_1\Omega_1\Omega_1\mat{W}^\top\right) +\rho_1 \left(-\sigma_1^2\Omega_1\Omega_1\mat{W} + \sigma_1^2\rho_1\Omega_1\Omega_1\mat{W}^\top\mat{W}\right) \label{eq:zVuZ_rho1_2}
\end{align}

Thus equation \ref{eq:zVuZ_rho1_1} can be rewritten as:

\begin{align*}
\mat{Z}\mat{V}_u\mat{Z}^\top
&=\mat{Z}
\Biggr[
\sigma_1^2\left(\begin{matrix}
\Omega_1\Omega_1 -\rho_1\Omega_1\Omega_1\mat{W}^\top & \mat{0}_{D\times DT}\\
\mat{0}_{DT\times D} &  \mat{0}_{DT\times DT}
\end{matrix}\right)\\
&+\rho_1\left(\begin{matrix}
-\sigma_1^2\Omega_1\Omega_1\mat{W} + \sigma_1^2\rho_1\Omega_1\Omega_1\mat{W}^\top\mat{W} & \mat{0}_{D\times DT}\\
\mat{0}_{DT\times D} &  \mat{0}_{DT\times DT}
\end{matrix}\right)\\
&+\sigma_2^2\left(\begin{matrix}
\mat{0}_{D\times D} & \mat{0}_{D\times DT}\\
\mat{0}_{DT\times D} & \Omega_2
\end{matrix}\right)
\Biggr]
\mat{Z}^\top \\
&=\left(\begin{matrix}\mat{Z}\bar{\Omega}_{1, \sigma_1^2}\mat{Z}^\top & 
\mat{Z}\bar{\Omega}_{1, \rho_1}\mat{Z}^\top & \mat{Z}\bar{\Omega}_2\mat{Z}^\top\end{matrix}\right)
\left(\begin{matrix}
\sigma_1^2 \\
\rho_1 \\
\sigma_2^2
\end{matrix}\right)
\end{align*}

Thus equation \ref{eq:rml_theta_fp} can be rewritten (analogously as
above) to:

\begin{eqnarray*}
\psi(\mat{r})^\top\mat{U}^{\frac{1}{2}}\mat{V}^{-1}\frac{\partial\mat{V}}{\partial\theta_l}\mat{V}^{-1}\mat{U}^{\frac{1}{2}} \psi(\mat{r}) = \tr\left(K\mat{V}^{-1}\frac{\partial\mat{V}}{\partial\theta_l} (\mat{Z}\mat{V}_u\mat{Z}^\top)^{-1} \left(\begin{matrix}\mat{Z}\bar{\Omega}_{1, \sigma_1^2}\mat{Z}^\top & 
\mat{Z}\bar{\Omega}_{1, \rho_1}\mat{Z}^\top & \mat{Z}\bar{\Omega}_2\mat{Z}^\top\end{matrix}\right)
\left(\begin{matrix}
\sigma_1^2 \\
\rho_1 \\
\sigma_2^2
\end{matrix}\right)\right)
\end{eqnarray*}

Let

\[
\left(\begin{matrix}
\psi(\mat{r})^\top\mat{U}^{\frac{1}{2}}\mat{V}^{-1}\frac{\partial\mat{V}}{\partial\sigma_1^2}\mat{V}^{-1}\mat{U}^{\frac{1}{2}} \psi(\mat{r}) \\
\psi(\mat{r})^\top\mat{U}^{\frac{1}{2}}\mat{V}^{-1}\frac{\partial\mat{V}}{\partial\rho_1}\mat{V}^{-1}\mat{U}^{\frac{1}{2}} \psi(\mat{r}) \\
\psi(\mat{r})^\top\mat{U}^{\frac{1}{2}}\mat{V}^{-1}\frac{\partial\mat{V}}{\partial\sigma_2^2}\mat{V}^{-1}\mat{U}^{\frac{1}{2}} \psi(\mat{r})
\end{matrix}\right)
= a(\theta) \text{ ,}
\]

then

\[
\theta = \left(\begin{matrix}
\sigma_1^2 \\
\rho_1 \\
\sigma_2^2
\end{matrix}\right) = A(\theta)^{-1} a(\theta) \text{ ,}
\] where

\begin{align*}
A(\theta) = \left(\begin{matrix}
\tr\left(\gamma(\sigma_1^2) \mat{Z}\bar{\Omega}_{1,\sigma_1^2} \mat{Z}^\top \right) & \tr\left(\gamma(\sigma_1^2) \mat{Z}\bar{\Omega}_{1,\rho_1} \mat{Z}^\top \right) &
\tr\left(\gamma(\sigma_1^2) \mat{Z}\bar{\Omega}_2\mat{Z}^\top \right) \\
\tr\left(\gamma(\rho_1) \mat{Z}\bar{\Omega}_{1,\sigma_1^2} \mat{Z}^\top \right) & \tr\left(\gamma(\rho_1) \mat{Z}\bar{\Omega}_{1,\rho_1} \mat{Z}^\top \right) &
\tr\left(\gamma(\rho_1) \mat{Z}\bar{\Omega}_2\mat{Z}^\top \right) \\
\tr\left(\gamma(\sigma_2^2) \mat{Z}\bar{\Omega}_{1,\sigma_1^2} \mat{Z}^\top \right) & \tr\left(\gamma(\sigma_2^2) \mat{Z}\bar{\Omega}_{1,\rho_1} \mat{Z}^\top \right) &
\tr\left(\gamma(\sigma_2^2) \mat{Z}\bar{\Omega}_2\mat{Z}^\top \right)
\end{matrix}\right)
\end{align*}

and
$\gamma(\theta_l) = K\mat{V}^{-1}\frac{\partial\mat{V}}{\partial\theta_l} (\mat{Z}\mat{V}_u\mat{Z}^\top)^{-1}$

\subsubsection{More on the Fixed Point}\label{more-on-the-fixed-point}

Inspired by: Chatrchi Golshid (2012): Robust Estimation of Variance
Components in Small Area Estimation, Master-Thesis, Ottawa, Ontario,
Canada: p.~16ff.:

\begin{quote}
The fixed-point iterative method relies on the fixed-point theorem: "If g(x) is 
a continuous function for all $x \in [a; b]$, then $g$ has a fixed point in $[a;
b]$." This can be proven by assuming that $g(a)\geq a$ and $g(b)\leq b$. Since 
$g$ is continuous the intermediate value theorem guarantees that there exists a 
$c$ such that $g(c) = c$.
\end{quote}

Starting from equation \ref{eq:rml_theta} where $\theta = \sigre$ we can
rewrite the equation such that:

\begin{align}
\label{eq:rml_theta_fp}
\Phi(\theta) = \aFH - \tr\left(K\mat{V}^{-1}\frac{\partial\mat{V}}{\partial\theta} (\mat{Z}\mat{G}\mat{Z}^\top)^{-1} (\mat{Z}\mat{G}\mat{Z}^\top)\right) =& 0
\end{align}

Note that because the matrix $\mat{R}$ is assumed to be known for the FH
model, it can be omitted. Note that under the simple Fay-Herriot Model
$\mat{Z}\mat{G}\mat{Z}^\top = \sigre\mat{I}$, where $\mat{I}$ is a
$(D \times D)$ identity matrix. Furthermore
$\frac{\partial\mat{V}}{\partial\theta} = \mat{I}$. Thus equation
\ref{eq:rml_theta_fp} can be rewritten to:

\begin{align*}
\psi(\mat{r})^\top\mat{U}^{\frac{1}{2}}\mat{V}^{-1}\mat{V}^{-1}\mat{U}^{\frac{1}{2}} \psi(\mat{r}) = \tr\left(K\mat{V}^{-1}\mat{G}^{-1} \sigre \right)
\end{align*}

This can be solved for the fixed Point and is directly presented in
algorithmic notation, such that: \[
\theta^{m+1} = A(\theta^{(m)})^{-1} a(\theta^{(m)}) \text{ ,}
\] where \[
A(\theta) = \tr\left(K\mat{V}^{-1}\mat{G}^{-1} \right)
\] and
\[a(\theta) = \psi(\mat{r})^\top\mat{U}^{\frac{1}{2}}\mat{V}^{-1}\mat{V}^{-1}\mat{U}^{\frac{1}{2}} \psi(\mat{r})
\]
