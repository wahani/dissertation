\section{Area Level Models}\label{area-level-models}

\subsection{The Fay Herriot Model}\label{the-fay-herriot-model}

The model is introduced by \textcite{Fay79} and is used in small area
estimation for research on area-level. It is build on a sampling model:
\[
\directStat_{\indexDomain} = \trueStat_{\indexDomain} + \samplingError_{\indexDomain},
\] where $\directStat_{\indexDomain}$ is a direct estimator of a
statistic of interest $\trueStat_{\indexDomain}$ for an area
$\indexDomain$ with $\indexDomain = 1, \dots, \nDomains$ and $\nDomains$
being the total number of areas. The sampling error
$\samplingError_{\indexDomain}$ is assumed to be independent and
normally distributed with known variances $\samplingVarianceIndexed$,
i.e.
$\samplingError_{\indexDomain}|\trueStat_{\indexDomain} \sim \mathit{N}(0, \samplingVarianceIndexed)$.
The model is modified with the linking model by assuming a linear
relationship between the true area statistic $\trueStat_{\indexDomain}$
and some auxiliary variables $\xArea$: \[
\trueStat_{\indexDomain} = \xArea^\top \beta + \randomEffectIndexed\text{, } \indexDomain=1,\dots, \nDomains.
\] Note that $\xArea$ is a vector containing area-level (aggregated)
information for $\nRegressor$ variables and $\beta$ is a vector
($1\times \nRegressor$) of regression coefficients describing the
(linear) relationship. The model errors $\randomEffectIndexed$ are
assumed to be independent and normally distributed, i.e.
$\randomEffectIndexed \sim \mathit{N}(0, \randomEffectVariance)$
furthermore $\samplingErrorIndexed$ and $\randomEffectIndexed$ are
assumed to be independent. Combining the sampling and linking model
leads to:

\begin{align}
\label{eq:FH}
\directStatIndexed = \xArea^\top \beta + \randomEffectIndexed + \samplingErrorIndexed.
\end{align}
